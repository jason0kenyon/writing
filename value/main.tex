%%%%%%%%%%%%%%%%%%%%%%%%%%%%%%%%%%%%%%%%%
% Diaz Essay
% LaTeX Template
% Version 2.0 (13/1/19)
%
% This template originates from:
% http://www.LaTeXTemplates.com
%
% Authors:
% Vel (vel@LaTeXTemplates.com)
% Nicolas Diaz (nsdiaz@uc.cl)
%
% License:
% CC BY-NC-SA 3.0 (http://creativecommons.org/licenses/by-nc-sa/3.0/)
%
%%%%%%%%%%%%%%%%%%%%%%%%%%%%%%%%%%%%%%%%%

%----------------------------------------------------------------------------------------
%	PACKAGES AND OTHER DOCUMENT CONFIGURATIONS
%----------------------------------------------------------------------------------------

\documentclass[11pt]{diazessay} % Font size (can be 10pt, 11pt or 12pt)
\usepackage{amsmath, amsthm}
\usepackage{indentfirst}
\newtheorem{thm}{Theorem}[section]
\newtheorem{cor}{Corollary}[section]
\newtheorem{defn}{Definition}[section]
\newtheorem{lem}{Lemma}[section]
\newtheorem{prop}{Proposition}[section]
\newtheorem{excs}{Exercise}[section]
\numberwithin{equation}{section}
%----------------------------------------------------------------------------------------
%	TITLE SECTION
%----------------------------------------------------------------------------------------

\title{\textbf{On the Continuum of Value and Dynamics Thereof} \\ {\Large\itshape}} % Title and subtitle

\author{\textbf{Jason Kenyon} \\ \textit{Joe Mama's}} % Author and institution

\date{\today} % Date, use \date{} for no date

%----------------------------------------------------------------------------------------

\begin{document}

\maketitle % Print the title section

%----------------------------------------------------------------------------------------
%	ABSTRACT AND KEYWORDS
%----------------------------------------------------------------------------------------

%\renewcommand{\abstractname}{Summary} % Uncomment to change the name of the abstract to something else




%----------------------------------------------------------------------------------------
%	ESSAY BODY
%----------------------------------------------------------------------------------------

\section*{Introduction}
The question of value is of the most indelible and enduring to date, and will likely continue as such, indefinitely. And as it so happens, it is of the most perplexing, yet, also, important. 
This claim is our primary axiom: \emph{the question of value is an important one}. If the reader seeks the verification of it here, he will be disappointed; any deductive system requires a foundation,
and here begins ours. 

Henceforth, all of our axioms will be in \emph{italic text} for clarity. It should also be mentioned that 
this work is a rigorous theory of the nature of reality---metaphysics, epistemology, morals, and logic. Accordingly, by axiom, it is meant in the purest sense, as in mathematics; and the bulk of the material here will be analagous to a mathematical proof.  

Although the importance of our problem is an axiom, it should be clear to the reader why: 
\textbf{value is the comprehensive \underline{scalar} quantity assigned to an object (\underline{tangible} or intangible), as a means by which we may effectively \underline{orient} ourselves within, and \underline{interface} with, the \underline{world}}.
\footnote{Definitions will be denoted by bold text; undefined terms will be denoted by underlined text.} \footnote{Additionally, we must assume that the English language and interpretation thereof is sound and sufficiently expressive for our needs. Thus terms deemed unclear will be considered undefined, while others, implicit.}

With our first axiom and definition, we will proceed to define 
the terms above and conclude one's ideal orientation and interface with reality. Furthermore, we will construct a system to apply our deduced ideal. Along the way, we will have recourse to further axioms; however, we aim to keep them as reasonable and few as possible.

\section*{The Environment; Embedded Systems}

To investigate value, we must first examine the system in which it is embedded---the universe. 


%------------------------------------------------


%----------------------------------------------------------------------------------------

\end{document}
