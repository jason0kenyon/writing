\documentclass{article}

\usepackage[utf8]{inputenc}
\usepackage{amsfonts}
\usepackage{amsthm}
\usepackage{amssymb}
\usepackage{amsmath}

\newtheorem{thm}{Theorem}[section]
\newtheorem{cor}{Corollary}[section]
\newtheorem{defn}{Definition}[section]
\newtheorem{lem}{Lemma}[section]
\newtheorem{prop}{Proposition}[section]
\newtheorem{excs}{Exercise}[section]
\numberwithin{equation}{section}
\title{Topology Motivation}
\author{Jason Kenyon}
\date{26 August 2022}
\begin{document}
    \maketitle
    \section{Introduction}
    \section{Number Systems}
    \begin{defn}
        The set $\mathbb{N}$ is the set such that $\emptyset \in \mathbb{N}$
        and for all $n \in \mathbb{N}$, $\emptyset \cup n \in \mathbb{N}$.
    \end{defn}
    Notice that the axiom of infinity justifies our assumption that $\mathbb{N}$ does indeed exist.
    \begin{defn}
        The set $\mathbb{Z}$ is the set $\mathbb{N}$ combined with an element, called the additive inverse, $z^{-}$ for each element $z^{+} \in \mathbb{N}$. This $z^{-}$ satisfies the property that
        for each $z^{+} \in \mathbb{N}$, there corresponds a $z^{-}$ such that $z^{-}+z^{+}=0$.
    \end{defn}
    Hence, by definition, the set $\mathbb{Z}$ is a totally ordered group under addition. 
    \begin{defn}
        The set $\mathbb{Q}:=\{\frac{p}{q}: \ p,q \in \mathbb{N} \ \mathrm{and} \ \mathrm{CD}(p,q)=1 \}$
    \end{defn}
    It is clear that the rational numbers are well-defined. The difficult next step is to 
    make precise what we know as real numbers; that is, how do we fill in the gaps of $\mathbb{Q}$?
    We begin with the formal notion of an ordering relation. Additionally, we see that, again, $\mathbb{Q}$ is a linearly ordered group under addition.
    \begin{defn}
        A partial ordering $\leq$ on a set X is a relation on X satisfying:
        \begin{enumerate}
            \item $x \leq x$ for all $x \in X$
            \item $x \leq y$ and $y \leq x$ implies that $x=y$ for all $x,y \in X$
            \item $x \leq y$ and $y \leq z$ implies that $x \leq z$ for all $x, y, z \in X$
        \end{enumerate}
        Furthermore, if, in addition to the above, we have that only $x \leq y$ or  $y \leq x$ or $x=y$,
        we say that the partial ordering $\leq$ is a linear partial ordering.
    \end{defn}
    This definition clearly encapsulates the intuitive ideas we have about order. For one cannot be less than two and equal to two
    simultaneously. The primary aim now is to demonstrate that our real numbers are unique, ordered, and operable---we can add and multiply.
    In other words, we want to show that $\mathbb{R}$ is the unique linearly ordered field such that any subset of $\mathbb{R}$ has a least upper bound. This is how
    we "fill in the gaps" of $\mathbb{Q}$. To begin, we investigate what we mean by distance.
    \begin{defn}
        A metric space $(X,d)$ on a set $X$ with distance function $d$ is a set $X$ along with a function $d: \ X^{2} \to [0, \infty)$ satisfying the following:
        \begin{enumerate}
            \item $d(x, y)=0$ if and only if $x=y$ for all $x,y \in X$
            \item $d(x,y)=d(y,x)$ for all $x,y \in X$
            \item $d(x,z) \leq d(x,y) + d(y,z)$ for all $x,y,z \in X$
        \end{enumerate}
    \end{defn}
    Note that the notation $[0, \infty)$ could be misleading. We have yet to define real numbers; therefore, we interpret that set 
    as a half open interval of rational numbers, not real numbers. And once we have defined the real numbers rigorously, we may bootstrap our definitions
    without fear of poor reasoning. Now we may quantify what it means for elements of a metric space to \emph{approach}
    others.



    \begin{defn}
        A sequence $a_n$ is a function $a_n: \ \mathbb{N} \to X$ for some set $X$.
    \end{defn}
    \begin{defn}
        A sequence $a_n$ is a Cauchy sequence if $$\forall \epsilon \in \mathbb{Q}( \epsilon>0 \implies (\exists N \in \mathbb{N} \ \forall n, m \in \mathbb{N} (n, m \geq N \implies d(a_n, a_m)< \epsilon)))$$
    \end{defn}
    \begin{defn}
        A sequence $a_n$ converges to $L \in \mathbb{Q}$ if  $$\forall \epsilon \in \mathbb{Q}( \epsilon>0 \implies (\exists N \in \mathbb{N} \ \forall n \in \mathbb{N} (n \geq N \implies d(a_n, L)< \epsilon)))$$
    \end{defn}
    The previous two definitions, as you can see, are nearly the same. A Cauchy sequence is a sequence whose outputs 
    become arbitrarily close; a convergent sequence is a sequence whose outputs tend toward some value. There ought to be some connection here; indeed,
    we call a metric space \emph{complete} if any Cauchy sequence converges.
    \begin{thm}
        If a sequence $a_n$ of rational numbers converges to $L$, then $a_n$ is a Cauchy sequence.
    \end{thm}
    \begin{proof}
        Suppose that $a_n$ is a convergent rational sequence. Let $\epsilon>0$. We know that, $|a_n-L|<\frac{\epsilon}{2}$ provided that $n \geq N$ for some $N \in \mathbb{N}$. If, in addicting, we have that for $m \geq N$ for an $m \in \mathbb{N}$, we have
        \begin{align}
            |a_n-a_m|&=|(a_n-L)-(a_m-L)| \\
            &\leq |a_n-L|+|a_m-L| \\
            &< \frac{\epsilon}{2} + \frac{\epsilon}{2} \\
            &= \epsilon
        \end{align}
        which is to say that $a_n$ is Cauchy.
    \end{proof}
    One can also prove the converse; that is, $(\mathbb{R}, d)$ is a complete metric space, where d is the euclidean norm.
    \begin{thm}
    Any rational Cauchy sequence $a_n$ is bounded (there exists some $M \in \mathbb{N}$ such that $a_n \leq M$ for all $n \in \mathbb{N}$). 
    \end{thm}
    \begin{proof}
        If $a_n$ is a rational Cauchy sequence, we know that there is an $N \in \mathbb{N}$ such that when $n,m \geq N$, then $ |a_n-a_m|< 1$. Clearly, then,
        $a_{N+1}-1<a_n<a_{N+1}+1$. Taking $M=\mathrm{max}\{|a_0|, |a_1|, \cdots |a_n|, |a_{N+1}+1|, |a_{N+1}-1| \}$, we find our bound.
    \end{proof}

    \begin{defn}
        Let $a_n$ and $b_n$ be rational Cauchy sequences. Then we define $a_n \approx b_n$ if the sequence $a_n-b_n$ converges to $0$.
    \end{defn}
    It should be clear that $\approx$ is an equivalence relation; that is,
    it is symmetric, reflexive, and transitive.
    \begin{defn}
        The real numbers $\mathbb{R}$, the equivalence classes $[a_n]$ where the $a_n$ are rational Cay sequences.
    \end{defn}
    Now we define addition and multiplication. Afterward, we will show that $\mathbb{R}$ is a field.
    \begin{defn}
        Let $a,b \in \mathbb{R}$. Then:
        \begin{enumerate}
            \item $a+b:=[a+b]$
            \item $a\cdot b:=[a \cdot b]$
        \end{enumerate}
        \end{defn}
        To convince yourself that the above is well defined, notice that 
        the sum and product of any two Cauchy sequences must necessarily also be Cauchy. And that our equivalence relation $\approx$
        is closed under addition and multiplication. Now, let us verify that $\mathbb{R}$ forms an ordered field.
        \begin{thm}
            Let $a \in \mathbb{R}$ be nonzero. Then there exists an $a^{-1} \in \mathbb{R}$ such that $a \cdot a^{-1}=1$
        \end{thm}
        \begin{proof}
            Suppose that $[a_n]$ does not converge to zero. Let $N$ be such that $a_n\neq 0$ for all $n\geq N$
            Define the rational Cauchy sequence $(b_n):=(0, 0, \cdots \frac{1}{a_N}, \frac{1}{a_{N+1}}, \cdots)$. Clearly, $a_n \cdot b_n=(0, 0, \cdots, 1, 1, \cdots)$, which converges to 1. Therefore, $[a_n \cdot b_n]=1$.
        \end{proof}
        It is trivial to verify that $\mathbb{R}$ also contains an additive inverse for each element.
        Associativity, commutativity, and distributively follow directly from our definition of $\mathbb{R}$ and that $\mathbb{Q}$ is a linearly ordered field.
        \begin{thm}
            Let $a,b,c \in \mathbb{R}$. If $a \leq b$ then $a+c \leq b+c$.
        \end{thm} 
        This fact is verified by, again, using our knowledge of $\mathbb{Q}$ and the transitivity of $\leq$ Theron.
        One can also verify that 
        \begin{thm}
            If $a,b \in \mathbb{R}$ and $a>0$ and $b>0$, then $a \cdot b > 0$
        \end{thm}
        And most importantly 
        \begin{thm}
            If $u \subseteq \mathbb{R}$ is nonempty and $u$ has an upper bound, then $u$ has a least upper bound.
        \end{thm}
        The proof will not be shown here, nor will the proof of uniqueness, for the sake of brevity.
        However, the reader should be able to convince themselves of this with an extrapolation from $\mathbb{Q}$ and
        our results on Cauchy sequences from before. The details are not necessarily important; what matters now is that
        we can safely work with the familiar field of real numbers, just as we have hitherto done, without the additional rigor.
    \section{Topology}
    With the real numbers in our toolkit, we can now investigate limits of functions from $\mathbb{R}^{n}$ to $\mathbb{R}$, what open sets are,
    and, of course, why topology is the way it is.
    \begin{defn}
        Let $X$ be a set. A topology $\mathcal{T}$ on $X$ is a collection of subsets of $S$ satisfying:
        \begin{enumerate}
            \item $X \in \mathcal{T}$
            \item If $I$ is an arbitrary index set and $a_i \in \mathcal{T}$ for all $i \in I$, then $\bigcup \limits_{i \in I}a_i \in \mathcal{T}$
            \item If $a,b \in \mathcal{T}$ then $a \cap b \in \mathcal{T}$
        \end{enumerate}
        Furthermore, we call the tuple $(X,\mathcal{T})$ a topological space on $X$.
    \end{defn}
    The question is why? Why should a topology be closed under arbitrary unions and finite intersections?
    To answer these questions, we use metric spaces. In particular, metric spaces entirely motivate
    what is done in topology; topology is but a fruitful abstraction from the familiar metric space structure. When we are no longer constrained by 
    geometric concerns, we are more able to answer some important questions about 
    the structure of sets, or geometric objects if you prefer.
    In analysis, we are familiar with the following notion of openness:
    \begin{defn}
        An open rectangle in $\mathbb{R}^{n}$ for an $n \in \mathbb{N}$ is a set $R \subset \mathbb{R}^{n}$ of the form
        $$R=\prod\limits_{i=1}^{n} \{x \in \mathbb{R}: a_i<x<b_i \ \mathrm{where} \ a_i,b_i \in \mathbb{R} \} $$
        And a set is open if it is of the above form, or an arbitrary union of such sets.
    \end{defn}
    In the case where $n=1$, we have only open intervals. This is, then, a natural extension
    of an open interval with which we are acquainted.
    \begin{thm}
        A finite intersection of open rectangles is an open rectangle. Thus, open sets (in the metric sense) are closed under arbitrary unions and finite intersections.
    \end{thm}
    \begin{proof}
        Let $R_1=(a_1, b_1) \times (a_2, b_2) \times \cdots (a_n, b_n)$ and $R_2=(c_1,d_1) \times (c_2, d_2) \times \cdots (c_n, d_n)$ be open rectangles in $\mathbb{R}^{n}$. Suppose $R_1 \cap R_2 \neq \emptyset$. If $A \subseteq B$ or $B \subseteq A$, then $A \cap B=A$ or $A \cap B=B$, which are certainly open rectangles. If this is not the case, then
        the boundary of $A$ and the boundary of $B$ must necessarily intersect, forming another open rectangle $A \cap B$.
    \end{proof}
  %Metric topology from open balls= metric topology from open rectangles
  %neighborhoods and limit points
  %demorgan, preimage
  %continuity


\end{document}
