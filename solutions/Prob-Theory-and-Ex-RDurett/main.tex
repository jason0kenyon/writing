%%%%%%%%%%%%%%%%%%%%%%%%%%%%%%%%%%%%%%%%%
% fphw Assignment
% LaTeX Template
% Version 1.0 (27/04/2019)
%
% This template originates from:
% https://www.LaTeXTemplates.com
%
% Authors:
% Class by Felipe Portales-Oliva (f.portales.oliva@gmail.com) with template 
% content and modifications by Vel (vel@LaTeXTemplates.com)
%
% Template (this file) License:
% CC BY-NC-SA 3.0 (http://creativecommons.org/licenses/by-nc-sa/3.0/)
%
%%%%%%%%%%%%%%%%%%%%%%%%%%%%%%%%%%%%%%%%%

%----------------------------------------------------------------------------------------
%	PACKAGES AND OTHER DOCUMENT CONFIGURATIONS
%----------------------------------------------------------------------------------------

\documentclass[
	12pt, % Default font size, values between 10pt-12pt are allowed
	%letterpaper, % Uncomment for US letter paper size
	%spanish, % Uncomment for Spanish
]{fphw}

% Template-specific packages
\usepackage[utf8]{inputenc} % Required for inputting international characters
\usepackage[T1]{fontenc} % Output font encoding for international characters
\usepackage{mathpazo} % Use the Palatino font

\usepackage{graphicx} % Required for including images

\usepackage{booktabs} % Required for better horizontal rules in tables

\usepackage{listings} % Required for insertion of code

\usepackage{enumerate} % To modify the enumerate environment

%----------------------------------------------------------------------------------------
%	ASSIGNMENT INFORMATION
%----------------------------------------------------------------------------------------

\title{Selected Solutions to Probability Theory and Examples by Rick Durrett} % Assignment title

\author{Jason Kenyon} % Student name

\date{\today} % Due date

\institute{Binghamton University \\ Department of Mathematics} % Institute or school name

\class{Independent Study} % Course or class name

\professor{Professor David Biddle} % Professor or teacher in charge of the assignment

%----------------------------------------------------------------------------------------

\begin{document}

\maketitle % Output the assignment title, created automatically using the information in the custom commands above

%----------------------------------------------------------------------------------------
%	ASSIGNMENT CONTENT
%----------------------------------------------------------------------------------------

\section*{Question 1.1.1}

\begin{problem}
	Let $\Omega=\mathbb{R}$, $\mathcal{F}$ be the set of all subsets of $\mathbb{R}$ such that it or it's complement is countable. Let $P(A)=0$ if $A$ is countable and $P(A)=1$ if $A^{c}$ is countable. Then $(\Omega, \mathcal{F}, P)$ is a probability space.
\end{problem}


\subsection*{Answer}
We begin by showing that $\mathbb{F}$ is a sigma algebra. Suppose that $x \in \mathcal{F}$. Then either $x$ countable or $\mathbb{R} \backslash x$ is countable. In the first case,
$\mathbb{R} \backslash(\mathbb{R} \backslash x)=x \in \mathcal{F}$. 
In the latter, $\mathbb{R} \in \mathcal(F)$. Now suppose that
$x_1, x_2, \dots \in \mathcal{F}$. If at least one of $\mathbb{R} \backslash x_i$ is countable, then
so must $\bigcap_i\mathbb{R} \backslash x_i$. Otherwise, it must be that all $x_i$ are uncountable, in which case $\bigcup_i x_i$ is too. Now, to demonstrate that our probability measure is well-defined, it is clear that by definition $P(x)\geq 0$ for any set $x \in \mathcal{F}$. Additionally, supposing $x_1, x_2, \dots \in \mathcal{F}$ are disjoint sets
we have either $P(\bigcup_ix_i)=0$ or $P(\bigcup_ix_i)=1$. In the first case, we know that the union is countable, which implies that each of $x_i$ must be countable, lest our union be uncountable. This is to say $P(\bigcup_ix_i)=\sum_i0=\sum_iP(x_i)$. In the second case, we have that $P(\bigcup_ix_i)=1$. And notice that if $x_i$ are disjoint then only a single member of that collection can have an uncountable complement, for otherwise they could not be disjoint. This is to say that $P(\bigcup_ix_i)=1+0+0+ \dots=\sum_iP(x_i)$.

%----------------------------------------------------------------------------------------


%----------------------------------------------------------------------------------------

\end{document}
