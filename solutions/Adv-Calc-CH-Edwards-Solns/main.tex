%%%%%%%%%%%%%%%%%%%%%%%%%%%%%%%%%%%%%%%%%
% fphw Assignment
% LaTeX Template
% Version 1.0 (27/04/2019)
%
% This template originates from:
% https://www.LaTeXTemplates.com
%
% Authors:
% Class by Felipe Portales-Oliva (f.portales.oliva@gmail.com) with template 
% content and modifications by Vel (vel@LaTeXTemplates.com)
%
% Template (this file) License:
% CC BY-NC-SA 3.0 (http://creativecommons.org/licenses/by-nc-sa/3.0/)
%
%%%%%%%%%%%%%%%%%%%%%%%%%%%%%%%%%%%%%%%%%

%----------------------------------------------------------------------------------------
%	PACKAGES AND OTHER DOCUMENT CONFIGURATIONS
%----------------------------------------------------------------------------------------
\documentclass[
	12pt, % Default font size, values between 10pt-12pt are allowed
	%letterpaper, % Uncomment for US letter paper size
	%spanish, % Uncomment for Spanish
]{fphw}

% Template-specific packages
\usepackage{amsmath}
\usepackage[utf8]{inputenc} % Required for inputting international characters
\usepackage[T1]{fontenc} % Output font encoding for international characters
\usepackage{mathpazo} % Use the Palatino font

\usepackage{graphicx} % Required for including images

\usepackage{booktabs} % Required for better horizontal rules in tables

\usepackage{listings} % Required for insertion of code

\usepackage{enumerate} % To modify the enumerate environment

%----------------------------------------------------------------------------------------
%	ASSIGNMENT INFORMATION
%----------------------------------------------------------------------------------------

\title{Advanced Calculus of Several Variables by C.H Edwards: Selected Solutions} % Assignment title

\author{Jason Kenyon} % Student name

\date{\today} % Due date

\institute{Binghamton University \\ Department of Mathematical Sciences} % Institute or school name

\class{Independent Study} % Course or class name

\professor{David Biddle} % Professor or teacher in charge of the assignment

%----------------------------------------------------------------------------------------

\begin{document}

\maketitle % Output the assignment title, created automatically using the information in the custom commands above
\section*{Section 1}
%----------------------------------------------------------------------------------------
%	ASSIGNMENT CONTENT
%----------------------------------------------------------------------------------------

\section*{Question 1.2}

\begin{problem}
	Let $V$ and $W$ be subspaces of $\mathbb{R}^n$. Then $V \cap W$ is too. 
\end{problem}
%------------------------------------------------

\subsection*{Answer}

Let $v_i \in V \cap W$ for $1 \leq i \leq n $ We know that each of these vectors must be in both $V$ and $W$, whence any linear combination of vectors in $V$ is in $V$, and similarly, any linear combination of vectors in $W$ is in $W$. 
Thus, any linear combination involving $v_i$ must necessarily be in $V \cap W$
%----------------------------------------------------------------------------------------

\section*{Question 1.3}

\begin{problem}
	If $V$ and $W$ are subspaces of $\mathbb{R}^n$, then the set $V+W=\{v+w: \ v \in V \text{ and }  w \in W\}$ is too.
\end{problem}

%------------------------------------------------

\subsection*{Answer}

If $v_1+w_1 \in V+W$ then,since $V and W$ are closed under scalar multiplication,
$av_1+aw_1 \in V+W$ for any scalar $a \in \mathbb{R}$.
Furthermore, supposing $v_1 + w_1 \in V+W$ and $v_2 + w_2 \in V+W$ for $v_1, v_2 \in V$
and $w_1, w_2 \in W$, we have: 
	$$(v_1+w_1)+(v_2+w_2)=(v_1+v_2)+(w_1+w_2) \in V+W $$
because both $V$ and $W$ are closed under vector addition.

	
%----------------------------------------------------------------------------------------

\section*{Question 1.5}

\begin{problem}
	Let $D_0$ be the set of all real-valued differentiable functions on $[0,1]$
where $f(0)=f(1)=0$. Then $D_0$ is a vector space if we define scalar multiplication and vector addition in the natural way.
\end{problem}

%------------------------------------------------

\subsection*{Answer} 
If $f,g \in D_0$ then $f+g$ must necessarily have the same domain as its component functions 
and be differentiable. Additionally, adding any two such functions will give you another, and multiplying by scalars does as well. Thus, it is clear that $D_0$ is a vector space. 

However, if we, instead, have that $f(0)=0$ and $f(1)=1$ for any function $f \in D_0$
$D_0$ can not be a vector space, for if $f \in D_0$ and $a \in \mathbb{R}$, we have
$af(1)=a \neq 1$. 
And so $af(1) \notin D_0$.

%----------------------------------------------------------------------------------------
\section*{Section 2}
\section*{Question 2.5}
\begin{problem}
	Let $V_1, \dots V_k$ be a set of linearly independent vectors in a vector space V. 
	If $k<n=\text{dim} V$, then there exist vectors $V_{k+1}, \dots V_n$
	in $V$ such that $V_1, \dots, V_k, V_{k+1}, \dots V_n$ is a basis 
	for $V$.
\end{problem}

\subsection*{Answer} 
By Zorn's Lemma, we know that any vector space has a basis. Furthermore, by definition of dimension, we know that any set of linearly independent vectors in $V$ has at most $n$ elements. Thus, since a basis is a linear independent generating set of vectors, and such a set must necessarily contain as many elements as the dimension of the space, it is clear that the set 
$$\{V_1, \dots, V_k\} \cup \{V_{k+1}, \dots V_n\}$$
is a basis for $V$, where $\{V_{k+1}, \dots V_n\}$ is the unique set of 
remaining vectors such that the union with our original set constitutes a basis.

\section*{Question 2.6}
\begin{problem}
	The following two statements are equivalent:
	\begin{enumerate}
		\item If $\text{dim}V=n$, then each basis for $V$ has exactly$n$ vectors.
		\item If the system \begin{align*} a_{11}x_1 + \dots  a_{1n}x_n&=0  \\ &\vdots\\ a_{n1}x_1+ \dots a_{nn}x_n&=0   \end{align*} has only the trivial solution, then if the zero vector is replaced by $b=(b_1, \dots b_n)$ above, the system has a unique solution.
	\end{enumerate}
\end{problem}
\subsection*{Answer}
Suppose that the first statement is true. Now, if our system above has only trivial solution, then that means the vectors
\begin{align*}
     \begin{bmatrix}
           a_{11} \\
           a_{12} \\
           \vdots \\
           a_{1n}
         \end{bmatrix} \dots
     \begin{bmatrix}
	     a_{n1} \\
	     a_{n2} \\
	     \vdots \\
	     a_{nn}
     \end{bmatrix}
  \end{align*}
are linearly independent by definition. Furthermore, by assumption they must 
form a basis for $V$. Hence, we can uniquely express any vector
$b=(b_1, \dots b_n) \in V$ uniquely as a linear combination of the above linearly independent vectors, which is to say that there is a unique solution to the above system where the zero vector is replaced by $b$.

Conversely, assuming the second statement, it is clear that the vectors
\begin{align*}
     \begin{bmatrix}
           a_{11} \\
           a_{12} \\
           \vdots \\
           a_{1k}
         \end{bmatrix} \dots
     \begin{bmatrix}
	     a_{k1} \\
	     a_{k2} \\
	     \vdots \\
	     a_{kk}
     \end{bmatrix}
  \end{align*}
  form a basis for the space consisting of all vectors $b=(b_1, \dots b_k)$,
  by the same logic. Therefore if we suppose that $\text{dim}V=n$ and $k<n$
  then clearly there exist vectors in $V$ that are not generated by our linearly independent vectors because the dimension of our space is strictly less than $\text{dim}V$. Similarly, if $k>n$, our space will generate vectors that are not in $V$.
\section*{Question 2.7}
\begin{problem}
	Any two colinear vectors are linearly dependent. And any 3 coplanar vectors are 
	linearly dependent. 
\end{problem}
\subsection*{Answer}
	Any vector space of dimension $n$ can have at most $n$ linearly independent vectors. Therefore, along any line (a one dimensional vector space), there can exist at most one linearly independent vector, and within a plane, two. 
\section*{Section 3}
\section*{Question 3.5}
\begin{problem}
	If $\cdot$ is the standard inner product on $\mathbb{R}^n$, then 
	$X \cdot Y = \frac{1}{4}(|X+Y|^2-|X-Y|^2)$
for any two vectors $X, Y \in \mathbb{R}^n$.
\end{problem}
\subsection*{Answer}
Applying the definition of the Euclidean norm, we have \\
$\frac{1}{4}(|X+Y|^2-|X-Y|^2) = \frac{1}{4}((X+Y) \cdot (X-Y))$. And by definition of the dot product, we have \\
$\frac{1}{4}((X+Y) \cdot (X-Y)) = \frac{1}{4}((x_1+y_1)^2 + \dots + (x_n+y_n)^2 -((x_1-y_1)^2 + \dots (x_n-y_n)^2))$. 
Multiplying our terms, it is clear that we are left with \\
$$\frac{1}{4}(4x_1y_1 + \dots 4x_ny_n )=X \cdot Y$$
\section*{Question 3.6}
\begin{problem}
	Let $a_1, \dots a_n $ be an orthonormal basis for $\mathbb{R}^n$. If $x=s_1a_1+ \dots s_na_n$ and 
	$y=t_1a_1 + \dots t_na_n$, then $x \cdot y=s_1t_1+ \dots s_nt_n$. 
\end{problem}
\subsection*{Answer}
Suppose that $a_i=e_i+v_i$ for some $v_i \in \mathbb{R}^n$ for all $i\leq n$. We may do so due to closeure of vector addition. 
Furthermore, we know that $|a_i|=|e_i+v_i|=1$, which implies that the ith components of $x$ and $y$ are $s_i$ and $t_i$ respectively. Hence, by definition of the dot product, $x \cdot y = t_1s_1 + \dots t_1s_2$. 
\section*{Question 3.8}
\begin{problem}
	Orthogonalize the vectors $e_i'=(1,1, \dots, 1, \underbrace{0}_{\text{ith component}}, \dots, 0)$ 
\end{problem}
\subsection*{Answer}
It is trivial to show that $\frac{<e_{k+1}', e_k>}{<e_k,e_k>}=1$. Thus, apply the Graham-Schmidt orthogonalization process, we find
$$e_{k}=e_{k}'-e_{k-1}-\dots = (0,0, \dots, 0, \underbrace{1}_{\text{k+1th component}}, 0, \dots, 0)$$
\section*{Question 3.11}
\begin{problem}
	The vectors $\frac{1}{\sqrt{2\pi}}, \frac{cos(x)}{\sqrt{\pi}}, \frac{sin(x)}{\sqrt{\pi}}, \dots $
form an orthogonal basis for the space of continuous functions on $[-\pi,\pi]$.
\end{problem}
\subsection*{Answer}
Firstly, note that $\int_{-\pi}^{\pi}\frac{cos(nx)}{\sqrt{2\pi}}  \ dx = \int_{-\pi}^{\pi}\frac{sin(nx)}{\sqrt{2\pi}} \ dx = 0 $ as we are merely integrating sinusoids over a single period, and sinusoids attain an equal but opposite value for each value they attain within a period. Therefore, we know that the first vector is orthogonal to the rest. 
Now, 
\begin{align}
	\int_{-\pi}^{\pi}sin(nx)cos(mx) \ dx & = \frac{1}{2}\int_{-\pi}^{\pi}sin((n+m)x)+sin((n-m)x) \ dx \\ 
&=0 
\end{align}
due to the angle-sum identities for trigonometric functions, linearity of the integral,  and the same argument as before.
\begin{align}
\int_{-\pi}^{\pi}sin(nx)sin(mx) \ dx &= \frac{1}{2}\int_{-\pi}^{\pi}cos(nx+mx)-cos(nx-mx) \ dx	\\
&= 0  
\end{align}
provided that $n\neq m$ (otherwise, our expression is not defined). Similarly, we may deduce the same for $cos$.
\section*{Section 4}
\section*{Question 4.6}
\begin{problem}
	For a linear mapping $L:\mathbb{R}^n \to \mathbb{R}^n$, $|X|=|L(X)|$ if and only if 
	$L(X) \cdot L(Y) = X \cdot Y$ for all $X,Y \in \mathbb{R}^n$.

	
\end{problem}
\subsection*{Answer}
Applying the polarization identity, we have $X \cdot Y=\frac{1}{4}(|X+Y|^2-|X-Y|^2)$. Thus, by definition of the norm, $$X \cdot Y=\frac{1}{4}(<X+Y,X+Y>-<X-Y,X-Y>)=<L(X),L(Y)>$$ by multilinearity of the inner product. The converse is shown by applying the same manipulations, but in the opposite direction.
\section*{Question 4.8 and Question 4.9}
\begin{problem}
	Deduce the matrix of a reflection in terms of rotations. Show that a translation is a rotation.
	
\end{problem}
\subsection*{Answer}
Geometrically, it is trivial to show that a reflection is but a composition of rotations of equal magnitude. That is to say, $$ T(\alpha)=R^2(\alpha)$$ where $T$ is the reflection of a point about the line through the origin at angle $\alpha$ with the horizontal, and $R$ is the reflection of a point about the origin by $\alpha$. Additionally, it is easy to show that the matrix for a reflection $R(\alpha)=\begin{pmatrix} cos(\alpha) & -sin(\alpha)\\ sin(\alpha) & cos(\alpha) \end{pmatrix}$ 
Matrix multiplaction yields 
\begin{align}
	T(\alpha)&=R^2(\alpha) \\
	&=\begin{pmatrix} cos(\alpha) & -sin(\alpha)\\ sin(\alpha) & cos(\alpha) \end{pmatrix}^2 \\
		&=\begin{pmatrix} cos^2(\alpha)-sin^2(\alpha) & -cos(\alpha)sin(\alpha)-sin(\alpha)cos(\alpha)\\ sin(\alpha)cos(\alpha)+cos(\alpha)sin(\alpha) & -sin^2(\alpha)+cos^2(\alpha) \end{pmatrix} \\
		&=\begin{pmatrix} cos(2\alpha) & -sin(2\alpha)\\ sin(2\alpha) & cos(2\alpha) \end{pmatrix}
\end{align}
\section*{Section 7}
\section*{Question 7.2}
\begin{problem}
	The function $f(x,y,x)= x^2y+2xz^2$ is continuous at $(1,1,1)$ 
\end{problem}
\subsection*{Answer}
To begin, let $\epsilon>0$ and assume that $\delta \leq 1$. This implies that when $|(x,y,z)-(1,1,1)|< \delta$
$|x-1|<1$, $|y-1|<1$ and $|z-1|<1$. Furthermore $1<x,y,z<2$ $2<|x+1|<3$ $2<|y+1|<3$ and $2<|z+1|<3$. It then follows that:
\begin{align}
	|f(x,y,z)-3|&=|y(x-1)(x+1)+(y-1)+2x(z-1)(z+1)+2(x-1)| \\
	&\leq |y||x-1||x+1|+|y-1|+2|x||z-1||z+1|+2|x-1| \\ 
	&< 8|x-1|+|y-1|+18|z-1| \\
	&< 27\delta
\end{align}
Therefore, taking $\delta=min(1,\frac{\epsilon}{27})$ we are done.
\section*{Question 7.4}
\begin{problem}
	Let $f(x,y)=\frac{2x^2y}{x^4+y^2}$ except for when $x=y=0$, in which case $f(x,y)=0$. 
	Then $f(x,y)$ is continuous on any line through the origin, but $f(x,y)$ is not continuous in general. 
	
\end{problem}
\section*{Answer}
Define $\phi(t)=(t,at)$. To show that $f(x,y)$ is continuous on any line through the origin, it suffices to show 
$$\lim\limits_{t \to 0}f(\phi(t))=0$$
$f(\phi(t))=\frac{2at}{t^2+a^2}$ by definition, and provided that $a\neq 0$, we have this. However if we redefine $\phi(t)=(t,t)$, which is the case of horizontal line through the origin, 
$f(\phi(t))=\frac{t}{t^2+1}$. 
And this approaches $0$ as $t$ does.Next, define $\psi(t)=(t,t^2)$. Taking the limit $$\lim\limits_{t \to 0}f(\phi(t))$$ we find that it is 1, as 

$f(\phi(t))=\frac{2t^4}{t^4+t^4}$. This violates the continuity criterion that $\lim\limits_{(x,y) \to (a,b)}f(x,y)=f(a,b)$
\section*{Question 7.6}
\begin{problem}
	Let $D \subseteq \mathbb{R}^n$. If $a \in \partial D$ then $a \in D$ or $a$ is a limit point of $D$. 
\end{problem}
\subsection*{Answer}
This follows from the trivial fact that any set can either can contain elements or not. That is, either $\partial D \subseteq D$, in which case, $a \in D$(and is still a limit point), or $\partial D \not\subseteq D $, in which case $a\not\in D$ but is a limit point of A. 
\section*{Question 7.8}
\begin{problem}
	If $f:\mathbb{R}^n \to \mathbb{R}^m$ is continuous at $a$ and the sequence $\{a_n\}_{n=1}^{\infty}$ converges to $a$, then the sequence $\{f(a_n)\}_{n=1}^{\infty}$ converges to $f(a)$. 
\end{problem}
\subsection*{Answer}
Let $\epsilon>0$. Then for $n \in \mathbb{N}$ we have $\forall_{m\geq n} |a_m-a|<\epsilon_2$ for $\epsilon_2>0$. And since we know that $f$ is continuous at $a$, for all $m\geq n$ $|a_m-a|<\epsilon_2<\delta$ implies that $|f(a_m)-f(a)|<\epsilon$ . Therefore, for any $\epsilon>0$ there exists $n \in \mathbb{N}$ such that for all $m\geq n$ $|f(a_m)-f(a)|<\epsilon$.  
\end{document}
