\documentclass{article}
\usepackage{amsmath}
\usepackage{amsthm}
\newtheorem{defn}{Definition}
\newtheorem{thm}{Theorem}

\title{Multilinearity and the Determinant}
\date{\today}
\author{Jason Kenyon}

\begin{document}
\maketitle
\section{Multilinearity}
Here, we define the determinant as the unique multilinear function that is alternating, and zero when given the identity matrix.
We then deduce some familiar properties thereof. Finally, we investigate other familiar multilinear functions, as well as introduce sequilinearity---(multi)linear(ish) functions. 

Henceforth, arguments of functions will be assumed to be vectors, not scalars, of an arbitrary space over an arbitrary field,
unless we state otherwise.
\begin{defn}
$det(A)=\sum_{i=1}^{n}(-1)^iA_{ij}det(A_{*ij})$ where the matrix $A_{*ij}$ is the matrix found by eliminating the $i^{th}$ row and $j^{th}$ column of A.
\end{defn}
\begin{defn}
An n-linear function $f:X \to Y$ is a function such that for all $1\leq i\leq n$ 
$$f(x_1, \dots, ax+y, \dots, x_n)=af(x_1, \dots, x, \dots, x_n)+f(x_1,\dots, y,\dots,x_n)$$,
where $ax+y$ is the $i^{th}$ argument of $f$.
\end{defn}
Now that we have our primary object of concern, let's estalish the properties our function must have:
\begin{defn}
An n linear function $f:X \to Y$ is alternating if $$f(x_1, \dots, x_n)=0$$ whenever $x_i=x_{i+1}$ for $1\leq i\leq n$
	
\end{defn}
Furthermore, suppose that 

$f(x_1, \dots, x_n)=1$ whenever $\begin{pmatrix}
x_1 \\
\vdots \\
x_n
\end{pmatrix}=I$

\begin{thm}
Suppose $f$ satisfies the above properties. Let $A= \begin{pmatrix}
x_1 \\
\vdots \\
x_n
\end{pmatrix}$ and $B=\begin{pmatrix}
x_1 \\
\vdots \\ 
x_{i+1} \\ 
x_{i} \\ 
\vdots \\ 
x_n
\end{pmatrix}$
for $1\leq i\leq n$. Then $f(B)=-f(A)$. 
\end{thm}
\begin{proof}
By the alternating condition and multilinearity of $f$, we have 
\begin{align}
0&=f\begin{pmatrix}x \\ \vdots \\ x+y \\ x+y \\ \vdots \\ x_n \end{pmatrix} \\
&=f\begin{pmatrix} x_1 \\ \vdots \\ x \\ x \\ x_n\end{pmatrix} + f\begin{pmatrix} x_1 \\ \vdots \\ y \\ y \\ x_n\end{pmatrix} + f(B)+f(A) \\ 
&=f(B)+f(A)
\end{align}	
\end{proof}
\begin{thm}
Let $A=\begin{pmatrix}x_1 \\ \vdots \\ x_n\end{pmatrix}$ with $x_i=x_j$ for $1\leq i,j \leq n$. Then $f(A)=0$
	
\end{thm}
\begin{proof}
By the previous theorem, $-f\begin{pmatrix}x_1 \\ \vdots \\ x_i \\ x_i+1 \\ \vdots \\ x_n \end{pmatrix}=f\begin{pmatrix}x_1 \\ \vdots \\ x_j \\ x_i+1 \\ \vdots \\ x_i \\ x_n \end{pmatrix}=0$	
\end{proof}
\begin{thm}
Let $A=\begin{pmatrix}x_1 \\ \vdots \\ x_n	\end{pmatrix}$ and $B=\begin{pmatrix}x_1 \\ \vdots \\ x_i+ax_j \\ \vdots \\ x_n \end{pmatrix}$
for $1 \leq j \leq n$.
	
\end{thm}
\begin{proof}
By the previous theorem and multilinearity of $f$:
\begin{align}
f(B)&= f(A)+af\begin{pmatrix}x_1 \\ \vdots \\ x_j \\ \vdots \\ x_n\end{pmatrix} \\ 
&=f(A)
\end{align}.
\end{proof}
\begin{thm}
Let $A=\begin{pmatrix}x_1 \\ \vdots \\ x_n\end{pmatrix}$. If $rank(A)<n$, then $f(A)=0$.
	
\end{thm}
\begin{proof}
Since $rank(A)<n$ we may deduce $f(A)=f\begin{pmatrix}e_1 \\ \vdots \\ 0 \\ \vdots \\ 0\end{pmatrix}$ with zero vectors beginning at the $rank(A)^{th}$ row. Therefore, we may rewrite $$f(A)=\begin{pmatrix}e_1 \\ \vdots \\ e_{rank(A)+i} - e_{rank(A)+i} \\ \vdots \\ e_n-e_n \end{pmatrix}$$ for $0 \leq i \leq n-rank(A)$. By multilinearity, it is clear that we may decompose this into permutations $f(A)=\sum_{i}^{n-rank(A)}f\begin{pmatrix}e_1 \\ \vdots \\ e_{rank(A)+i} \\ \vdots \\ e_n \end{pmatrix}-f\begin{pmatrix}e_1 \\ \vdots \\  e_{rank(A)+i} \\ \vdots \\ e_n \end{pmatrix}$=0
	
\end{proof}
\begin{thm}
For elementary matrices $E_1, E_2, and E_3$ of type 1,2, and 3 respectively $f(E_1)=-1$, $f(E_2)=kf(E_2)$ \text{, and} $f(E_3)=1$.
\end{thm}
Each of the above follow directly from the multilinearity of $f$ and the assumption that $f(I)=1$. Hence, we shall omit the proof.
\begin{thm}
f(AB)=det(A)f(B).	
\end{thm}
\begin{proof}
We start by proving the theorem for elementary matrices then extend that to arbitrary ones. 
By the previous theorem, $f(E_1)=-1$ and $f(E_1A)=-f(A)$ by theorem 1. Thus, $f(E_1A)=det(E_1)f(A)$ since $E_1$ is the result of swapping a row in the identy matrix. Next, $f(E_2)=k$ by the previous theorem and $det(E_2)=kdet(I)=k$. Since $f(E_2A)=kf(B)$, we have $f(E_2A)=det(E_2)f(A)$. Finally, by the previous theorem, we have $f(E_3)=1$ and $det(E_3)=det(I)+kdet(E)$ for some matrix $E$ with two equivalent rows. Therefore, $f(E_3A)=f(A)=det(E_3)f(A)$.

We know that $rank(AB)\leq rank(A)$ by definition of matrix multiplication, and $f(A)=0 if rank(A)<n$ by theorem 4. Thus if $rank(A)<n$, 
$f(AB)=0$ and similarly, $det(A)=0$. So $f(AB)=det(A)f(B)=0$. 
If $A$ is full rank, then we may represent $A=E_1E_2\dots E_n$ as full rank implies that $A$ may be completely row reduced. This means that 
\begin{align}
f(AB)&=f(E_1\dots E_nB) \\ 
&=det(E_1)f(E_2\dots E_nB) \\ 
&= \dots \\
&=det(E_1)\dots det(E_n)f(B) \\ 
&=det(E_1\dots E_n)f(B) \\ 
&=det(A)f(B)
\end{align}
\end{proof}
\begin{thm}
f(A)=det(A)	
\end{thm}
\begin{proof}
If $rank(A)<n$ then $f(A)=0$ by theorem 4. Similarly, $det(A)=0$. 
Otherwise, just as before we may represent $A=E_1 \dots E_n$. Hence  
\begin{align}
f(A)&=f(E_1\dots E_n) \\
&=det(E_1)f(E_2 \dots E_n) \\
&= \dots \\ 
&=det(E_1)\dots det(E_n) \\ 
&=det(E_1\dots E_n) \\
&=det(A)
\end{align}
\end{proof}
Therefore we have shown that the determinant is nothing but the alternating, multilinear function that evaluates to 1 given the identity.
\end{document}
