\documentclass{book}

\usepackage[utf8]{inputenc}
\usepackage{amsfonts}
\usepackage{amsthm}
\usepackage{amssymb}
\usepackage{amsmath}

\newtheorem{thm}{Theorem}[section]
\newtheorem{cor}{Corollary}[section]
\newtheorem{defn}{Definition}[section]
\newtheorem{lem}{Lemma}[section]
\newtheorem{prop}{Proposition}[section]
\newtheorem{excs}{Exercise}[section]
\numberwithin{equation}{section}

\title{Goals}
\author{Jason Kenyon and Friends}
\date{August 2022}
\begin{document}
\maketitle
\tableofcontents

\chapter{Introduction}



\chapter{What is a Goal?}
\begin{defn}
A goal is a self-contained organizational unit comprising all that is necessary to enact change in a predictable manner. \footnote{Literally, a goal is nothing but a desired outcome, but for the sake of brevity and modularity, we take occasion to redefine the word. Besides, with common usage of \emph{goal}, implicit is the need for action and planning.} 
\end{defn}
With a goal, as we define it, you may rest assured that your desired outcome will be realized more or less the way you envisioned it. In fact, execution is the easy part; the difficulty lies in converting the imprecise visions we have into actionable, precise plans that result in the achievement of something approximating what you initially desired. This is the goal---no, aim, of this work.
\section{The Aim}
\begin{defn}
An aim is a specific change you want to realize in the world, coupled with the timeframe thereof.
\end{defn}
Now, how do we acquire an aim from our vision? First, we need a vision. That is, we need to spend quality time without distractions pondering, and most importantly, writing about who we are, who we want to be, and how we will get there. These prompts are nontrivial to answer and will very widely per person and per person per second; we are continuously transforming, be it regressing or progressing\footnote{We also must define what we mean by progress and regress; however, take whatever philosophy you may have and apply it here. Are you an evil nuclear warmonger? Great, then regression for you might be killing less people than you did last week. No? Well, good.}, yearning for improvement, attempting to achieve it, and then completely changing our direction and bounds for success as we traverse our path: we do not know what we want, why we want it, or how to get it, and once we get it---we don't know if it's really what we wanted to begin with or, if it is, that we were right to want it. Regardless, we must certainly be uncertain; thus, we will henceforth have recourse to probabilities and averages to define our parameters but will act with complete assurance, for achievement requires such arrogance despite the probabilistic reality. For further philosophical discussion, read the appendix.

\subsubsection{Constructing the Aim}
The procedure for constructing an aim is as follows:
\begin{enumerate}
    \item Be present, mindful, and clear-headed. Regularly take time to brainstorm.
    \item Regularly write out the vision of your life using the prompts above as a starting point.
    \item Using the above strategies, conceive an end that is meaningful(to you). Expound the bounds for success and a timeframe.
    \item Regularly repeat 1-3
\end{enumerate}
Notice that I say \emph{regularly}. This does not mean tell yourself, "Yes, I will do it \emph{regularly}." You must list out every minute of every day from the point at which you begin to the point at which you accomplish your aim. Regularly is not a helpful word for the unconscious mind; if you do it \emph{regularly}, you will do it once or twice per week for a couple of weeks and then never again. You need precis expectations for yourself; and that is another aim of this book: you may reference this at any point, and should, when  you are uncertain about the expectations for yourself. In fact, you should write your own book for each end you are pursuing, and add to it until you finish.

To elaborate on the procedure above, I we will leave mindfulness and mental health to the appendix, along with philosophy. We will discuss the rest now. This is a malleable protocol, but take this as a starting point: take time to brainstorm and clear your mind of intrusive thoughts by writing, with pen and paper. Quality is irrelevant; get your thoughts down, be it worries, interests, goals, etc. It does not matter. But do it. 

Next, consider making a vision for yourself on the first Sunday of each month. Ask yourself the aforementioned questions among other things. Stop when you feel done and agree that the person whom you are describing on paper corresponds to the person whom you'd like to be. Similarly, complete your brainstorm when you are out of ideas or have another obligation.

Finally, using your regular introspection---during which you most certainly would have thought about what you would like to do on this planet---try and describe, in one sentence, exactly what you want to do(not what you want). And don't forget to add a deadline for yourself, which, of course, may change and should change, but don't be afraid to be ambitious with your timelines; this will make you better as long as you don't allow failure to defeat you. In particular, you should welcome failure.\footnote{An entire appendix will be devoted to confronting and exploiting failure.}
\subsubsection{Achieving the Aim}
\begin{defn}
An action plan is a precise description of each action one must take in order to achieve an aim.
\end{defn}

To achieve an aim, we devise an action plan. What does this look like? Briefly, we break our aim down into sub-aims and our sub-aims into tasks---directly actionable steps taken toward achieving our aim.
\begin{defn}
A sub-aim is a component of an aim. Every aim is a sub-aim, but not all aims require multiple sub-aims. A task is an immediately executable operation in accordance with its respective sub-aim.
\end{defn}
In essence, all tasks are goals. They contain a plan (that is, do the thing), they contain an aim (that is, do the thing), and they contain a \emph{maintenance plan}(that is, throughout the process of completing a task, one must ensure that they are acting accordingly and responding to conflicts efficiently).Therefore, we may conceptualize each goal as a set $G=\{A, P, M\}$, where $A$ is the aim, $P$ is the action plan and $M$ is the maintenance plan. Furthermore,  for each sub-aim $S$ we have $S \subseteq G$ for some goal $G$ and for each task $T$ $T\subseteq S$ for some sub-aim $S$. The purpose of sub-aims and tasks is to modularize and simplify the process of bringing things into the world. Now we must consider the dynamics of both our aims and the antagonistic world around us.
\subsubsection{Maintaining the Aim}
\begin{defn}
A maintenance plan is a precise description of the actions one must take in order to adapt---or even abandon---one's goal.
\end{defn}

The procedure presented for constructing the aim is a sort of meta-maintenance plan. Accordingly, we will define: 
\begin{defn}
A meta-goal is a self-contained organizational unit used to precipitate a goal.
\end{defn}
Hence, we adopt the convention that for each goal $g$ and meta-goal $G$, we have $g\subseteq G$. From before, we know that brainstorming and mindfulness lead to the conception of desired outcomes. If we also require ourselves to schedule time to elaborate further on these desired outcomes, then we can be sure that we will eventually generate a goal, and later, provided it goes unabandoned, a product (that, in an ideal world, approximates our initial vision or adaptation thereof). We may conclude the following:
\begin{enumerate}
    \item To change, we must plan
    \item To plan, we must aim
    \item To aim, we must think
\end{enumerate}

Now, let's discuss how we handle dynamic and conflict. 

The method I propose is to regularly practice meta-cognition; that is, ask yourself "What is going well?" "What is going poorly?" and "How do can I make the poorly going go well?"\footnote{You should also be sure to analyze whether things are truly going well or not. This is an essential, nontrivial judgement call. If you are wrong, that is okay. If you don't know, that is okay, too.} And, of course, as before, you should write the answers to these questions and more on a regular basis. I recommend doing this each week for each concurrent goal you are pursuing.
Additionally, this meta-cognition should include a review of your goal---analyze your the statement of your aim, the action plan, and the maintenance plan. You should adjust each of them, taking into account your answers to the questions above. Again, this takes as long as it takes. If you know of better questions to ask yourself, add them. If you need to write more on a topic, do it. If you are certain of your goal and trajectory and it takes merely five minutes, that's fine too. 

In summary, each of our lives, in a literal sense, is a goal. Its aim is what you want to accomplish before you die. Its action plan, the routine on which we have here elaborated. And its maintenance plan,again, comprised by the above routine. 
\begin{thm}
If you adhere to an exact protocol including systematic brainstorming and goal construction, you will effectively approximate the flighty desired outcomes in your mind. Furthermore, you will be happier for it, for nothing supersedes the value of changing our world in our image.
\end{thm}
\chapter{The System}
The following is a template for applying the information we have hitherto discussed. The hope is that the reader will be able to 
tailor it to his needs; however, the principles are invariant in any case.
\section{Aim}
\subsection{Routine}
Firstly, we need a routine for getting our thoughts in order and our expectations met. 
Every day:
\begin{itemize}
    \item Sit in silence and write out any thoughts you have. Elaborate on the ones that interest you the most. Note what you wish to investigate more or take action on (15 min+)
\end{itemize}
Every Month:
\begin{itemize}
    \item Write in detail about the vision for your life. Who are you? Who do you want to be(30 min+)
\end{itemize}
And some habits:
\begin{itemize}
    \item Wake up at a fixed time.
    \item Avoid overstimulation and simple pleasures. 
    \item Breathe correctly and be present
    \item Exercise every day in some capacity. Have a physical goal.
    \item Eat things that were once living. Avoid processed meals. 
    \end{itemize}
\subsection{Metrics}
It is important to record measures of interest each day so that we may effectively tweak our system. Depending on what you are optimizing for, different metrics will be of interest consider the following:
\begin{enumerate}
    \item Hours slept
    \item Food consumed, caffeine ingested(or other, possibly illicit, substances) and when
    \item Deep work hours
\end{enumerate}
A mathematician should optimize for quality deep work hours. More quality deep work hours=more theorems.
\section{Scheduling and Time}
The specifics will vary from person to person and goal to goal. Depending on priority we may differentiate the degree to which we plan to plan and plan to do, of course.
\subsection{Time Management System}
Every day:
\begin{itemize}
    \item Use your brainstorm and goals to fill in your calendar.
    \item Apply the time-blocking method to plan your day by the hour
\end{itemize}
The in-depth descriptions of the tasks you need to carry out will be in the action plan; in the calendar and hourly schedule, we include short references to our respective goals' tasks. You should know precisely what is of expected of you; the point here is to figure out when. Implicitly, we will need a calendar, notebook, and pen. The simpler the better.

The daily plan, like any other, is dynamic and so you should edit the plan as conflicts arise or adaptations are made.

As the days pile on, aims will come to mind. This is your opportunity add a date on your calendar to elaborate on this aim. This is the birth of a meta-goal. When you reach a day on which you have to elaborate on an aim you want to pursue, develop a procedure to ensure that your goal is formalized and you are able to execute.
Note that oftentimes, and I find this convenient, meta-goals and goals absorb one another and become indistinguishable. This is clear in the weekly scheme to which I am partial. Suffice it to say that once you have a goal in motion, you should be aware of their differences but exploit the convenience of combining the two. We are optimizing for efficiency and quality, so are plans are doomed to be incomplete from the beginning, yet we want as much structure as allows us to not spend more time getting structured.

Thus I propose,
every week:
\begin{itemize}
    \item Ensure that you have a week's worth of executable tasks for each of your goals.
\end{itemize}
Each goal should include an all-encompassing action plan, but precision must taper down as the timeframe gets larger. Otherwise
you would be doing nothing but planning and not actually doing anything. A week in advance is not too much to ask. And yes, the plan will change each day.

The achievement of your goals follows directly from this routine procedure. I view this as this maintenance and action plan to the meta-goal of life itself. Find something that works, formalize it, and change it as needed.

I suggest using pen and paper to initially develop your goals and latex to format them with more precision. This way the ideas are not lost in the void of notebooks and paper and you can remain certain about the expectations you have for yourself. I recommend leaving this level of formality for goals with timeframes greater than a month and touching on only the big picture; you can leave weekly logistics to the notepad, but you should always have somewhere to look when you don't know what is next. On shorter time-scales, this is the notebook, the schedule, and the planner. On larger ones, this is the formal document outlining your goal. You can add to this every week after you do your meta-cognition, for example. As always, consult your priorities and decide whether this is a good fit for your goal.

The major theme here is that we need to have routine tasks on our schedules that ensure that we get what we want. The regularity with which such things are done prevents stagnation and uncertainty. Without these boring, logistical demands on ourselves, nothing will happen; that is how people are. To summarize, use regular thoughtfulness and structure to stimulate consistent and desired outcomes.
\subsection{Diversions and Avocations}
It's good to have something to which you are only partially devoted on the side of your main pursuit. You will find that when one goes poorly, the odds of the other going poorly are low, and so the quality of life is generally better. However, such things consume time, so if you are serious about making something happen, this should be minimized.
\subsection{Chores}
Being a person not only requires that we make real progress on what is important to us, but also that we bathe ourselves, wash the dishes, and check in with our family and friends. Anything that is not pertinent to a current goal of yours is a chore. All such things should be systematized and, whenever possible, outsourced. Additionally, you should create an environment conducive to the least amount of energy-sapping errand running as possible. How can you expect to create anything if you are occupied with cleaning or emailing or fussing with your dining room table. None of these things matter. Your brain may, and likely will, tell you otherwise; you must not listen, for your potential will be entirely diminished.
Some practical examples of this are:
\begin{itemize}
    \item Do not own a home unless you can afford someone to keep it for you
    \item Buy regular household products such as soap or toothpaste in bulk and at regular intervals
    \item Contract a cleaner every time your living space becomes morbidly unsanitary
\end{itemize}
\chapter{Applications}


\section{Studying Mathematics}
Rigorous deep work protocol is essential. All time spent doing mathematics must be in deep work blocks---that is, in chunks of 15 min to 2hr work continuously, then take a break for 15 min to 1hr. An iterative approach is required to digest mathematical material. First, gain a cursory overview of the topic from a source like wikipedia or \emph{The Companion}.At the same time, amass a list of resources on you will use on the topic. Skim the main works you aim to cover to get a better feel for the topic.
 Next, begin a surface level study of the material. That is, work through the main theorems, do lots of problems, etc. The goal here is to understand precisely why, but not necessarily how, the present subject is the way it is. Finally, review the material with the lens of a total skeptic; you should already know why you are proving what you are proving, and now you want to go through the minutiae. Continue working through problems, preferably tougher ones. Now the end point, although it may vary, should often lie in constructing the theory on your own, from motivation, to definitions, to proofs. However, this is quite the task and clear boundaries should be set as to the extent to which you wish to do this. With this in mind, you should clarify how far you wish to go; it is also important to note that this will change with time. As long as it's thought through beforehand you should be fine. In some cases, however, you may find yourself aimlessly toiling away due to the sheer breadth and beauty of mathematics. Avoid this at all costs. Make sure you have reasonable, clear boundaries(this is not very helpful as reasonable and clear are relative terms, but stay aware of this trap. It is typically easy to identify once you have been there before.)
Be sure to work with the appropriate speed depending on the level of study you are on. If you are on a maximally deep dive of integration, for example, you should take months just to understand what volume means. Again, this is all imprecise, but we must work with averages and uncertainty. This is the real world. Just try not waste your precious time on trivialities. 
\section{Financial Freedom}
\subsection{Introduction}
Financial freedom can be most fundamentally defined as the state in which one has the ability to do WHAT they want, WHEN they want to. How difficult the barrier to financial freedom is to breach is highly dependent on the individual in question. 

There are a number of variables that need to be taken into consideration when one is planning for their eventual financial independence. I will list these variables, then elaborate further on each of them. They are, in no particular order, the following:
\begin{itemize}
    \item Living Costs
    \item Current Income
    \item Financial Liabilities
    \item Market Conditions
\end{itemize}
\subsection{Living Costs}
Notice that I distinctly separate living costs from financial liabilities. The term 'living costs' in this context refers to any and all costs that are necessary for one to continue living. For sake of simplicity, this encompasses any and all food purchases, housing payments, as well as other utilities such as heating, air conditioning, electricity, running water, etc. 

This also covers any and all expenses that are mandated by the state or federal government, such as income tax, property tax, etc. 

It is paramount that one has at least a rudimentary understanding of their living costs. Living costs are arguably the most important variable when it comes to securing one's financial independence, next to financial liabilities (as we will see).
\subsection{Current Income}
All else held equal, a higher income does indeed help secure one's financial independence more quickly. However, it is by no means the end all be all in terms of projecting one's potential to become financially free. 

If one cannot cut living expenses or their financial liabilities, they should naturally look to increasing their income. Instead of focusing on 'getting a raise' or something of that effect, diversify your income streams via passive income business models. Only a handful of quality products are truly necessary in order to bring in a sustainable, passive income. 

In today's information society, it is easier than ever to get paid. This is even more applicable in terms of automated services wherein you have the ability to receive what is known as passive income. Examples of sources of passive income would be something such as a website on which an e-book is sold, or a course on some subject you have expertise in. Passive income is the economic exchange in which the vendor (who is selling the product being bought) needs very little (or minimal) oversight in order for the exchange to occur successfully. 

When a passive-income stream is implemented and regularly brings in profit, time will no longer be as strongly correlated to income. As opposed to working for an hourly rate with an employer, you now have the potential for much greater income. Obviously, there comes with this the risk of a less-than-adequate income to live off of. In the past, this was a necessary risk. Today, it is possible to cultivate a passive-income whilst also maintaining a full-time job. That goes without saying that this all begins with some passion, and the willingness to create.

Those looking to eventually make the switch from active to passive income should not test the depth of the river with both feet. Instead, it is encouraged that one continue working a full-time job until they are able to live off of their passive-income stream (as in it covers both living expenses and financial liabilities).

\subsection{Financial Liabilities}
\subsection{Market Conditions}
\appendix
\chapter{Tools and Strategies}
\section{Deep Work}
Single task, effective mind. Breaks.
\section{Mindfulness}

\section{Minimalism}
\subsection{Single-purpose Technology}
Pens, paper, timers, forks, cups.


\chapter{Mental and Physical Health}
We have chosen to distinguish mental and physical out of convenience for the reader, but the two could not be less distinct. It is hardly fair to categorize problems as \emph{mental} and others as \emph{physical}. What matters, however, is that without health, you can not perform. By leaving underlying health concerns in the background you are (likely unwittingly) divesting yourself of the most highly and broadly effectual state humanly achievable. Nothing is more productive for an ill person than becoming healthy.  
\section{Mental Health}
\subsection{Depression}
\subsection{Anxiety}
Anxiety is a broad class of mental disorders centered about a common feeling of severe apprehension. These are no trivial worries; they are pervasive, chaotic, and profound. They produce lasting states of discomfort and pain, affecting every aspect of one's life. Chronically, they can shorten your life, and certainly make its shorter duration unbearable. 

Most people are anxious but are completely unaware. It is easy to miss the omens when you lack the awareness of their existence. Let's begin with identifying it then: you will notice a heightened heart-rate, irregular breathing, (and possibly uncontrollable shaking or jitters). The difficult part is that in such states, often you lose what little alertness you have and enter the solitude of your scattered mind: you do not hear or feel or taste; simply you observe one increasingly radical thought follow the last until---all of the sudden---you are back in the world again. You must practice midfulness and catch your every departure. This is the starting point.

Next, you have to reach for any of the thoughts, which previously grabbed you, that you can vividly recall. After this, it is nothing but categorization: classify each thought as either rational or imaginary, and then, each rational thought as relevant or irrelevant to you. Often our problems are not problems at all. That is, they may be completely impossible (or highly unlikely) or we may have no direct control over them. The former category is quite simple, but the latter can be subtle at times. For example, you can do your best to treat others with respect but can't expect them to reciprocate; this is no fault of yours and nothing you can do will exert control over others (we can barely control ourselves. I mean, you're reading a life manual. Yikes).
Another important example of this nuance is demanding that your body feel a certain way or sleep for some specific time: unfortunately, unconscious processes such as sleep and digestion are not within our direct control, and so it is completely illogical to set expectations on such things. The body is an immensely complex dynamical system with somewhat of a \emph{conscious} "control system" that doesn't know how to control itself, and even if it did, wouldn't know what changes it should make. We are horrible bosses and worse employees, yet we need to be both if we are to build something worth the air we breathe in our lifetimes.
\section{Physical Health}
\subsection{Consumption and Drugs}
It's likely to best to avoid most illicit substances, especially those that inhibit neurophysiological ability. Drugs like alcohol and marijuana---acceptable \emph{fun} drugs---are of particular importance for us. As society has accepted such things as righteous, or at least not condemnable, it is necessary that we make sure not to use that to justify wasting away like everyone else. I am not saying you should not have a good time---no, you should; in fact, if you are going to indulge, you best do it to the highest degree. I am saying that habitual consumption of anything that inhibits you mentally or physically is just silly. And it's very easy to be convinced, by yourself or others, that there is no difference between your sober output and your high output: there is.
Stimulants, however(caffeine or amphetamines), when taken at the proper doses provably and clearly exhibit remarkable increases in creative output. Some psychedelics have been shown to provide increased creativity as well.(Insert more information on the literature. I genuinely have never read about this)** 
Regardless, be sure to consider your goals whenever deciding to begin or continue any habit, be it drug related or other. 
\subsection{Sleep}
A flagrant and overwhelmingly common issue, especially for high-achievers, is sleeping well and sleeping consistently. It is also more likely than not that if you are depressed and or anxious that you also have trouble sleeping.(Whether you sleep poorly because you are mentally ill or are mentally ill because you don't sleep well is irrelevant. Furthermore, it's likely that both of these are causing your problem.) As I mentioned before in the section on anxiety, sleep is an unconscious bodily function, the machinations of which have been developed over millions of years for the express purpose of forgetting its existence. And now we want to \emph{hack our sleep} and \emph{fall asleep in two minutes guarantees} and all of the medical \emph{professionals} tell us that we need \emph{eight hours}. If you want to sleep, I'd say that's fairly normal and fine, but if you can't sleep and want to sleep, that is also fairly normal but not fine. If you can't sleep, then you can't \emph{need eight hours} or \emph{fall asleep now}. Just as if you eat a large meal, you can't \emph{be starving instantly}. We have no direct control over our bodily functions (including thinking), but we can, with what little conscious liberty we salvage, leverage those completely deterministic things to chronically influence our future selves: we can't stop being depressed, but we can start getting some sunlight and exercising; we can't sleep more, but we can work on our anxiety so that we don't have copious amounts of adrenaline flowing through our bodies precisely when we would like to rest. These things are done without notable expectations and over extended periods of time. Furthermore, as if to add to the paradox we already have, can not become our primary focus, lest we become obsessed and yet worse off. Exert direct influence on what you can control, don't lose your conviction, and move forward.

Now, since we are most definitely high-achieving people, and much awaits us each morning, we create the perfect environment for worries about sleeping well: we want to as much energy as possible so we can execute each of our tasks with tact and ease; of you do not have this desire, then you should probably read something else. Tomorrow is unfulfilled potential; it defines the human experience in a literal sense. Everything we do (beyond residual animalistic tendencies from evolution.) is for tomorrow. If we plan our week, we'll have a much easier and certain path toward success. If we do the hard (and efficient) work now, we will be rewarded later. What, then, is one to do if they want nothing more than their body to automatically perform its duties? Well, for one, get healthy, but beyond that, nothing. And to get healthy is neither specific nor a completed task; it is a consistent, demanding nuisance that, when ignored, will topple you. No one action will make a difference, but one action for twenty seconds of each hour of each day for a decade, will. 
Therefore, if you struggle to sleep, I recommend that you not turn of all the lights in the house by 5:00pm, that you not meditate and inhale incense to the sound of rain drops hitting sheet metal; I want you to find the brightest light you can and look directly at it; make sure to set an early alarm for the following day; and, of course, carefully examine the clock while you are at it. "Oh no," your mind will tell you, "you had to be asleep an hour ago. You have so many responsibilities, and you didn't sleep the night before either. You will definitely fail!" Welcome that voice. Use it as an opportunity learn about yourself and better understand the routine, uncontrollable processes of the body. At the very least, you must know that ignoring all of this and praying that you do get sleep is not only futile but quite dull; it's much more interesting to shock your primitive self than yield to its influence and \ldots pray?
\subsection{Exercise}
You need  a physical hobby.


\chapter{Philosophy}
\section{Logic}
\subsection{Uncertainty}

\section{Metaphysics}
\section{Ethics}


\chapter{Failure}
\section{Exploiting Failure}

\end{document}

